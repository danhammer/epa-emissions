% Created 2012-10-26 Fri 16:53
\documentclass{article}
\usepackage[utf8]{inputenc}
\usepackage[T1]{fontenc}
\usepackage{fixltx2e}
\usepackage{graphicx}
\usepackage{longtable}
\usepackage{float}
\usepackage{wrapfig}
\usepackage{soul}
\usepackage{textcomp}
\usepackage{marvosym}
\usepackage{wasysym}
\usepackage{latexsym}
\usepackage{amssymb}
\usepackage{hyperref}
\tolerance=1000
\usepackage{color}
\usepackage{minted}
\usepackage{mathrsfs}
\usepackage{graphicx}
\usepackage{amstex}
\usepackage{booktabs}
\usepackage{dcolumn}
\usepackage{subfigure}
\usepackage[margin=1in]{geometry}
\RequirePackage{fancyvrb}
\DefineVerbatimEnvironment{verbatim}{Verbatim}{fontsize=\small,formatcom = {\color[rgb]{0.1,0.2,0.9}}}
\providecommand{\alert}[1]{\textbf{#1}}

\title{}
\author{}
\date{\today}
\hypersetup{
  pdfkeywords={},
  pdfsubject={},
  pdfcreator={Emacs Org-mode version 7.8.02}}

\begin{document}



\setlength{\parindent}{0in}
\renewcommand{\X}{{\bf X}}
\renewcommand{\uab}{\bar{U}_A}
\renewcommand{\ubb}{\bar{U}_B}
\renewcommand{\xao}{X_{A1}}
\renewcommand{\xat}{X_{A2}}
\renewcommand{\xbo}{X_{B1}}
\renewcommand{\xbt}{X_{B2}}
\renewcommand{\L}{\mathscr{L}}
\renewcommand{\st}{\hspace{8pt} \mbox{s.t.} \hspace{6pt}}
\renewcommand{\y}{{\bf y}}

\textbf{EPA light-duty vehicle emission description} \hfill
Dan Hammer \\ \\

\section*{Emissions standards (and useful life)}
\label{sec-1}


The following description taken from \href{http://www.dieselnet.com/standards/us/ld_t2.php}{this site}. \\

The Tier 2 emission standards are structured into 8 permanent and 3
temporary certification levels of different stringency, called
“certification bins”, and an average fleet standard for NOx
emissions. Vehicle manufacturers have a choice to certify particular
vehicles to any of the available bins. When fully implemented in 2009,
the average NOx emissions of the entire light-duty vehicle fleet sold
by each manufacturer has to meet the average NOx standard of 0.07
g/mi. The temporary certification bins (bin 9, 10, and an MDPV bin 11)
with more relaxed emission limits are available in the phase-in period
and expire after the 2008 model year.

Tier 2 vehicles are those meeting the requirements of one of the
available bins and that are used to meet the requirement that a
percentage of the fleet have average NOx emissions of 0.07
g/mile. During the phase-in period, the rest of the fleet not used to
comply with the 0.07 g/mile NOx average are referred to as interim
non-Tier 2 vehicles. They must still meet the requirements of one of
the available bins but have more relaxed fleet average requirements.

The emission standards for all pollutants (certification bins) when
tested on the Federal Test Procedure (FTP) are shown in Table 2. Where
intermediate useful life exhaust emission standards are applicable,
such standards are applicable for five years or 50,000 miles,
whichever occurs first. The vehicle “full useful life” period for LDVs
and light LDTs has been extended to 120,000 miles or ten years
whichever occurs first. For heavy LDTs and MDPVs, it is 11 years or
120,000 miles whichever occurs first. Manufacturers may elect to
optionally certify to the Tier 2 exhaust emission standards for
150,000 miles to gain NOx credits or to opt out of intermediate life
standards. In such cases, useful life is 15 years or 150,000 miles,
whichever occurs first. For interim non-Tier 2 LDV/LLDTs, the useful
life is 10 years or 100,000 miles, whichever occurs first.

\end{document}
